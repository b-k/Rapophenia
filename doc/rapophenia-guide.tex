\documentclass{article}
\usepackage{xspace,url}
\begin{document}

\author{Rolando A Rodr\'iguez and Ben Klemens}
\title{Rapophenia}
\def\ad{{\tt apop\_data}\xspace}
\def\am{{\tt apop\_model}\xspace}
\maketitle

Rapophenia is an R interface to the Apophenia library for statistical
computing. Apophenia provides many conveniences at roughly the same level as R,
like functions to fit OLS models, stack and subset data sets, calculate singular
value decompositions, et cetera. When you want your C code to drive R graphics, or
when your R program needs to run an order of magnitude faster so you rewrite it in C,
Rapophenia will smooth the transition.

Apophenia is built on two basic objects: the \ad structure, which is
analogous to R's data frame, and the \am, which is intended to standardize
the expression of several different types of model. Thus, Rapophenia provides utilities
for sending these two structures across the R|C border.

The R interface has only two C interface functions to work with, and their arguments
must be one of four types ({\tt int*}, {\tt double*}, {\tt char**}, or {\tt SEXP}).
Rapophenia provides a way to get through the bottleneck, because you can now copy a
full data frame or \ad across the R|C border, retaining most of the structure (such as
column and row names).

Although every statistical model one could imagine is implemented in an R package, there
is no standardization across packages, so it is difficult to replace one model with
another. Although there is extra work that goes into getting a model to conform with a
standard model format, the payoff is easy interchangeability, even if you never send your
model to the C side. 

R is ill-suited for simulations, which will typically run an order of magnitude faster in
C. By writing the model in C and wrapping it in Rapophenia's model form, you will be able
to use R's statistical routines on simulation-style models.

\section{Installation} Rapophenia won't make much sense without installed copies of R and
Apophenia. Apophenia, in turn, depends on having development versions of the GNU
Scientific Library and the SQLite library. Installation instructions for these components
are easily found online.

\section{\ad} Most of Apophenia's functions are based on the \ad set, which is basically a
bundling of a {\tt gsl\_matrix} and {\tt gsl\_vector} (brought in from the GNU Scientific
Library), a grid of text, and a set of names. C is a typed language and works best when
items of like type are grouped together; this is effectively an R data frame given those
constraints.

For both data frame $\to$ \ad and \ad $\to$ data frame, all data is copied (not pointed
to or aliased).  R uses the FORTRAN-style column-major ordering, while C uses row-major
ordering, so we would need tricks to make non-copying work. Also, there's really no
way to register to R's garbage collection that a data set is also pointed to by a C
structure, so if we didn't copy the data, it could disappear at any moment.

Factors are not yet implemented, although both sides have pretty similar means of handling
them (R stores a list of category names in  a list of attributes; Apophenia stores them
by adding a page of data to the main data set). Because factors in R are implemented
as integers with labels added as attributes, they will appear in the matrix of the
\ad after translation.

\paragraph{R side} From an R data frame, use 
\begin{verbatim}
cptr <- apop_data_from_frame(r_frame)
\end{verbatim}
The function generates a C-side \ad, wraps it in an R object, and returns that for you to
save to whatever name. The R documentation
refers to the result as an {\em external pointer}, and other sources call this sort of
thing an {\em opaque pointer}. In R, all you can do with such a pointer is keep it on hand
and pass it back to later C functions that will make use of the same data set. If you are
using R as a control language that calls a series of C functions, then this will be
useful; there are also a few examples below.

To go the other way, from an R-wrapped external pointer to an \ad to a data frame, use:

\begin{verbatim}
r_frame <-data_frame_from_apop_data(cptr)
\end{verbatim}


\paragraph{C side} 
On the C side, every R element---data frames, vectors, external pointers, functions,
everything else---is wrapped in a {\tt SEXP} (an S expression pointer). 

If you find yourself in a block of C code with a {\tt SEXP} pointer to a data frame, then
use 

\begin{verbatim}
apop_data *dataset = apop_data_from_frame(SEXP_in);
\end{verbatim}

Now you can process your data using Apophenia and the GNU Scientific Library.

When you are done processing, if you need to return data to the R side, then 
produce an output data frame that can be returned to R via:

\begin{verbatim}
SEXP R_object_out = data_frame_from_apop_data(apop_data_in);
\end{verbatim}

Or just end your function with:
\begin{verbatim}
return data_frame_from_apop_data(apop_data_in);
\end{verbatim}


If you have used the R-side \verb@apop_data_from_frame@ to generate a pointer to an \ad
set, then you can unwrap the SEXP to get to the pointer itself via the R-standard:
\begin{verbatim}
apop_data *d = R_ExternalPtrAddr(input_SEXP);
\end{verbatim}


\section{\am} Apophenia's model structure provides a standardized form for models, so that
function authors can be guaranteed that input models will meet certain expectations. For
example, Apophenia has a maximum likelihood function that takes in any model with a log
likelihood or probability (and an optional dlog likelihood) and searches for the optimum
using a variety of methods. The optimization treats the log likelihood or probability
itself as a black box.

Apophenia has default methods to fill in when a slot in the structure is empty. For
example, if there is no routine to estimate the parameters of a model given input data,
but there is a log likelihood, then Apophenia will use the above-mentioned maximum
likelihood routine to derive the parameters; see the example below. If you provide a
method to make random draws from your model, Apophenia fills in the CDF method by
counting what percentage of random draws are below the input value.

\paragraph{R side models}
Rapophenia's models are S4 classes with slots that mirror the slots in Apophenia's models.
After initialization, you will be able to use any of the C side routines that take \am{}s
as inputs on models that you wrote in R. There are many steps to the process, some of
which will change in the next version.

The steps:

First, write the methods as R functions. Each R function takes in a single environment as
an argument (because we rely on a C side routine that evaluates a (function, environment)
pair). You can expect that there is an element in the environment named {\tt parameters},
which will be placed there by the evaluate routine. For example, here is Rosenbrock's
banana function. Notice that we also assume that there is a {\tt scaling} element in the
environment.

\begin{verbatim}
ll <- function(env){
    return (-((1-env$parameters$Vector[1])**2 +
    env$scaling*(env$parameters$Vector[2]-env$parameters$Vector[1]**2)**2))
}
\end{verbatim}

Second, write the model, which is a new object that puts your functions into the right
slots. There are also a few auxiliary elements, such as the {\tt vbase}, {\tt mbase1}, and
{\tt mbase2} elements, which specify the size of the parameter set, and the {\tt name}.
[ Apophenia limits the name to 100 characters; anything longer is truncated.]
The {\tt data} element is a list with whatever elements may be needed for evaluating the
model, and will be passed around with the model.

\begin{verbatim}
modobj <- new("apop_model",
    name="banana", ll_function=ll, data=data.frame(scaling=3), 
    vbase=2L, settings=setobj)
\end{verbatim}

Third, send the model object to the setup function, which will generate an \am structure
with appropriate pointers to your R-side functions, and return an external pointer to the
model. You now have a model with no parameters set (unless you explicitly set them in
the {\tt data} element). For example, out of the box the {\tt apop\_normal} distribution
has an undefined mean and variance.

At this point, you can (almost) throw out your S4 model, because all information has been
copied to the C side, and you will use the external pointer for all future steps. As long
as your S4 class is in scope, we can presume that R will not garbage-collect its elements
and the C side pointers will point to valid data; after your class goes out of scope,
you can expect that any uses of the C side model will segfault.

Fourth, estimate the model. In this step, you would send in data to find the best
parameters given the data. The returned model, with parameters, can then be used for
random draws, testing, or just to print the parameters.



\begin{verbatim}
mod <- setupRapopModel(modobj)
data <- as.environment(list(scaling=scale))
est <- estimateRapopModel(data, mod)
\end{verbatim}


\paragraph{C side models} Apophenia ships with a few dozen statistical models that you may
wish to call from R; see the full list at \url{http://apophenia.info/group__models.html}.
There are extra steps for setting up a new model that you have written yourself, below.

On the C side, Apophenia follows a principle that you shouldn't have to register your
models with a central database---just declare a new model and use it on the next line.
This falls apart when interfacing with R, because the R|C interface has no way to pull
up an arbitrary C symbol. Therefore, the first step is to tell C to initialize an {\em
ad hoc} registry:

\begin{verbatim}
.C("init_registry")
\end{verbatim}

This requirement is likely to be automated soon.

Now you can pull a model from the registry using its name. The name is easy to find in the
Apophenia documentation at the link above. Once you have the external pointer in hand, you
can use it almost as you would an R-side model. The only difference is that C side models
can do nothing with an R environment or list, so the first argument needs to be either
{\tt NULL} or an external pointer to an \ad, which you set up using one of the methods
described above.

\begin{verbatim}
mod <- get_C_model("OLS")
dataset <-apop_data_from_frame(data_frame)
est <- estimateRapopModel(dataset, mod)
params <- getModelElement(est, "parameters")
print(params)
\end{verbatim}

Given the facilities Apophenia provides, you may be able to write new models in C about
as easily as you can in R. See the Apophenia documentation at \url{http://apophenia.info}
for all the considerations that go into structuring a model and its procedures. 

To use a model with Rapophenia, you will need to register it in the list of available
models.

\begin{verbatim}
//Probably static (initialized on startup):
apop_model newmodel= {"A new model", .vbase=1, .p=..., .rng=...};

//called via .C("init_my_models"), 
//in lieu of the use of "init_registry" above.
void init_my_models(){
  init_registry();
  add_to_registry(&newmodel);
}
\end{verbatim}

The registry uses the model name as the identifier, so you will need a unique name for
every model.

\end{document}
